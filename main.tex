%----------------------------------------------------------------------------------------
%	PACKAGES AND OTHER DOCUMENT CONFIGURATIONS
%----------------------------------------------------------------------------------------
\documentclass[margin, centered]{res}
\topmargin=-0.5in
\oddsidemargin -.5in
\evensidemargin -.5in
\textwidth=6.5in
\itemsep=0in
\parsep=0in
\newsectionwidth{1in}
\usepackage[pdftex]{graphicx}
\usepackage{enumitem}
\usepackage{wrapfig}
\usepackage{helvet}
\usepackage{color}
\usepackage[colorlinks = true,
            linkcolor = blue,
            urlcolor  = blue,
            citecolor = blue,
            anchorcolor = blue]{hyperref}
\setlength{\textwidth}{6.5in} % Text width of the document
\setlength{\textheight}{720pt}

\begin{document}

%----------------------------------------------------------------------------------------
%	NAME AND ADDRESS SECTION
%----------------------------------------------------------------------------------------\
\begin{center}
    \hspace{-\hoffset}
    %\huge\bf{\href{https://www.github.com/PrajwalaTM}{PrajwalaTM}}
    \huge\bf{\href{https://www.linkedin.com/in/prajwala-tm-77305a108/}{Prajwala TM}}
\end{center}
\vspace{-7mm}
\moveleft\hoffset\vbox{\hrule width 19cm height 0.5pt}
\vspace{-9mm}
\begin{center}
    \hspace{-\hoffset}
    \href{mailto:prajwala.tm@gmail.com}{prajwala.tm@gmail.com} ~\textbullet~ \(+91\) 8970679533 ~\textbullet~ \#90, Sneha Sadan,Banashankari 3rd stage, Bangalore, India
\end{center}
\vspace{-7mm}
\begin{resume}

%----------------------------------------------------------------------------------------
%	EDUCATION SECTION
%----------------------------------------------------------------------------------------
\section{Education}
\textbf{B.Tech in Computer Engineering} \hfill 2014 - 2018 \\
\href{http://nitk.ac.in/}{National Institute of Technology Karnataka, Surathkal, India}
\begin{itemize}
 \item CGPA of \textbf{9.01}/10
\end{itemize}
\textbf{Pre-University Board} - {Deeksha Center For Learning} - 94.16\% \hfill 2012 - 2014 \\
\textbf{Higher Secondary School (CISCE) } - Little Flower Public School - 96.57\% \hfill 2010 - 2012
 
%----------------------------------------------------------------------------------------
%	EXPERIENCE SECTION
%----------------------------------------------------------------------------------------
\section{Experience}
\textbf{Software Engineer, \href{http://www.arcesium.com/}{Arcesium India Pvt. Ltd}} \hfill July, 2018 - Present \\
Working as a Software developer at \textbf{Arcesium} (former part of D.E. Shaw Group), for a team that automates the management reporting processes. Currently, I am working on a new project that has been kickstarted in the team for the migration of the existing setup to a new web application that provides enhanced data insights to the users. Apart from the development of the web application, I also played a major role in the design, configuration setup and the evaluation of the choice of tools for the project. \\
\textbf{Tools} - Kotlin, Java8, React.js with Mobx, Spring, Maven, MyBatis, SQL Server,
Perl\\ \\
\textbf{Software Developer Intern, \href{http://www.arcesium.com/}{Arcesium India Pvt. Ltd}} \hfill May, 2017 - July, 2017\\
Worked on a software that portrayed the Arcesium platform as a
unified searchable repository, built on top of other platform applications. Major tasks involved implementation of a data extractor module, and persistence of polled data in the database. The services written were exposed RESTfully and the project was moved to the first stage of production during the course of the internship. \\
\textbf{Tools} - Java8, Spring, Maven, MyBatis, PostgreSQL, TestNG, Flyway\\
\\
\textbf{Summer Intern,  \href{http://www.iitd.ac.in/}{IIT Delhi}} \hfill May, 2016 - July, 2016 \\
\emph{Mentored by \href{http://www.cse.iitd.ac.in/~srsarangi/}{Dr. Smruti Ranjan Sarangi , Dept. of CSE, IIT Delhi}} \\
Worked on the solution manual for 3 chapters in the text book "Computer Organisation and Architecture - Smruti Ranjan Sarangi" 
Simulated a microprocessor for \textit{SimpleRisc} assembly language in Logisim\\
\textbf{Tools} - \LaTeX{},Logisim 

\textbf{FOSSEE Winter Intern, \href{http://www.iitb.ac.in/}{IIT Bombay}} \hfill Dec, 2015\\
Designed a toolbox named \href{http://www.scilab.in/scilab-toolbox-help-files/fgoalattain.php}{fgoalattain} for Scilab, that solves a multi-goal attainment problem\\
\textbf{Tools} - Scilab

%---------------------------------------------------------------------------------------
%	PUBLICATION SECTION
%---------------------------------------------------------------------------------------

\section{Publications}
%\begin{itemize}[leftmargin=*]
%\item
Guru Pradeep Reddy T, Kandiraju Sai Ashritha, \textbf{Prajwala T M}, Girish G N, Abhishek R. Kothari, Shashidhar G Koolagudi and Jeny Rajan, \\
{``Retinal Layer Segmentation using Dilated Convolutions''}, \href{http://www.iiitdmj.ac.in/CVIP-2018/}{2018 Third International Conference on Computer Vision and Image Processing}, Proceedings in \textbf{Springer} \\ \\
%\item
%\item
\textbf{Prajwala TM}, Alla Pranathi, Kandiraju Sai Ashritha, Nagaratna B. Chittaragi*, Shashidhar G. Koolagudi, \\
{``Tomato Leaf Disease Detection Using Convolutional Neural Networks''}, \href{http://www.jiit.ac.in/jiit/ic3/home.html}{Proceedings of 2018 Eleventh International Conference on Contemporary Computing (IC3), 2-4 August, 2018, Noida, India} \\ \\
%\item
%\item
Kandiraju Sai Ashritha, \textbf{Prajwala TM}, K Chandrasekaran,\\ \href{https://link.springer.com/chapter/10.1007/978-3-319-62698-7_1}{``Activity Theory Based Approach for 
Requirements Analysis of Android Applications''}, \href{http://www.kmo2017.com/}{International Conference on Knowledge Management in Organizations}, Beijing, 2017, Proceedings in \textbf{Springer CCIS} - 10.1007/978-3-319-62698-7 
%\item
%\end{itemize}

%----------------------------------------------------------------------------------------
%	TECHNICAL SKILLS SECTION
%----------------------------------------------------------------------------------------
\section{Technical \hspace{2mm} Skills}
\textbf{Languages} - Java, Kotlin, Javascript, SQL, Python, Perl, C, C++ (STL)\\
\textbf{Tools/Frameworks} - React.js, Spring, Maven, Mybatis, Django, Keras, PostgreSQL, \LaTeX, R, Git

%----------------------------------------------------------------------------------------
%	RELEVANT COURSE SECTION
%----------------------------------------------------------------------------------------
\section{Relevant \hspace{2mm} Courses}
Data Structures and Algorithms, Software Engineering , Artificial Intelligence and Neural Networks, Machine Intelligence, Computer Architecture, Operating Systems, Computer Networks, Database Management Systems , Computer Vision, Distributed Computing, Number theory and Cryptography, Internet Technologies and applications, Compiler Design

%----------------------------------------------------------------------------------------
%	Selected Projects Section
%----------------------------------------------------------------------------------------
\section{Projects}
All projects available on \href{https://github.com/PrajwalaTM}{GitHub}
%\setlist[itemize]{
\begin{itemize}[leftmargin=*]
\item \textbf{\color{blue}{Retinal Layer Segmentation using Deep Neural Networks}} \hfill Oct, 2017 - April, 2018\\
\emph{Mentored by \href{http://cse.nitk.ac.in/faculty/jeny-rajan}{Dr. Jeny Rajan , Dept. of CSE, NITK, Surathkal}}\\
The project aims to segment the retinal layers of OCT images using deep learning neural networks. It involves comparison of existing deep neural net approaches and building a novel architecture inspired by Dilated Convolutional networks for the semantic segmentation of the same. The project has been issued by MHRD, Govt. of India. 
\item \textbf{\href{https://github.com/PrajwalaTM/FCN-based-Semantic-Image-Segmentation}{Semantic Object Segmentation using FCN}} \hfill Oct 2017\\
Used Fully Convolutional Networks and VGG Net model for semantic segmentation of images from PASCAL-VOC dataset. Implementation was carried out using Keras.

 \item \textbf{\color{blue}{Plant Leaf Disease Detection}} \hfill Feb, 2017 - Apr, 2017\\
 \emph{Mentored by \href{http://cse.nitk.ac.in/faculty/shashidhar-g-koolagudi}{Dr. Shashidhar G Koolagudi, Dept. of CSE, NITK, Surathkal}}\\
 A MATLAB application which identifies the diseases in tomato leaves based on machine learning techniques. GLCM has been used for feature extraction and SVM,Random Forest algorithms have been used for multi-class classification and a comparitive analysis has been carried out.\\
 LeNet and Wide Residual Network deep learning architecture implementation has given good results. 
 \item \textbf{\href{https://github.com/PrajwalaTM/Raft-Consensus-Algorithm}{Raft}} \hfill Oct, 2016 - Nov, 2016\\
 \emph{Mentored by \href{http://cse.nitk.ac.in/faculty/annappa}{Dr. Annappa, Dept. of CSE, NITK, Surathkal}} \\
 Implemented Raft , a distributed consensus algorithm in \textbf{GO}. The main features include implementation of leader election,log replication and safety. The implementation minimizes the RPC count, handles race conditions, network failures and partition of network, without any compromise on the consensus.
 \item \textbf{{\href{https://github.com/PrajwalaTM/Hospital-Home}{Hospital-Home}}} \hfill Oct, 2016 - Nov, 2016\\
 \emph{Mentored by \href{http://cse.nitk.ac.in/faculty/p-santhi-thilagam}{Dr.P Santhi Thilagam, Dept. of CSE, NITK, Surathkal}}\\
 A Django application that performs disease prediction based on the symptoms input using Natural Language Processing Techniques. Finding the nearest doctor and heart disease prediction have also been implemented using GeoCoding of postal codes and Random forest algorithm respectively.
 \item \textbf{\href{https://github.com/PrajwalaTM/Android-Projects/} {Restaurant Automation}} \hfill Feb, 2016 - Apr, 2016\\
  \emph{Mentored by \href{http://cse.nitk.ac.in/faculty/k-chandrasekaran}{Dr. K. Chandrasekaran, Dept. of CSE, NITK, Suratkal}}\\
 An android application which automates the operations in a restaurant for improvement of performance, accuracy, customer experience and reduction of manual error. Android Studio, MySQL and PHP has been used for the same.
 \item \textbf{\href{https://github.com/PrajwalaTM/Text-Mining/tree/master/WikiArticlesClassification}{Wikipedia Articles Classification}} \hfill Oct, 2016 - Nov, 2016\\
 K-means clustering and Topic modelling using Latent Dirichlet Allocation algorithm for the categorization of Wikipedia articles.
 \item \textbf{\href{https://github.com/PrajwalaTM/Friendbook}{Friendbook}}  \hfill Oct, 2016 - Nov, 2016\\
 A semantic based friend recommendation system for social networks based on lifestyles. The lifestyles are extracted as life documents using Latent Dirichlet Allocation algorithm, and a friend-matching graph is constructed based on the same. Impact ranking, inspired by PageRank algorithm is used for prediction of recommendation scores. The implementation has been carried out using R.
\end{itemize}

%----------------------------------------------------------------------------------------
%	ACHIEVEMENT SECTION
%----------------------------------------------------------------------------------------
\section{Positions of Responsibility}
\begin{itemize}
\item Placement Co-ordinator of Computer Engineering at NITK, Suratkal for the year 2017-18
\item Treasurer of Spicmacay, NITK
\item Executive Member of WebClub, NITK.
\end{itemize}

\section{Achievements}
\begin{itemize}
 \item Pre-placement offer at Arcesium India Private Ltd, India. 
 \item Secured Rank 35 in CET 2014 Engineering Entrance Exam
 \item Secured Rank 3 in ComedK Engineering Entrance Exam
 \item Secured 273/360 in JEE Mains 2014, AIR 2479
 \item Secured AIR 3834 in JEE Advanced 2014
\end{itemize}

\section{Activities}
 Competitive coding, reading tech blogs, playing Guitar

\end{resume}
\end{document}